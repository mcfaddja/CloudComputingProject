\documentclass{article}[10pt]
\setlength{\textheight}{8.75in}
\setlength{\topmargin}{-.25in}
\setlength{\oddsidemargin}{-.5in}
\setlength{\evensidemargin}{0in}
\setlength{\textwidth}{7.25in}
\usepackage{amsfonts, amsmath, amsthm, amssymb,mathrsfs}
\usepackage{graphicx}
\usepackage{fancyhdr}
\usepackage{setspace}
\usepackage{xcolor}
\usepackage{mathtools}
\usepackage[]{algorithm2e}
\usepackage{algorithmicx}
\usepackage{multicol}
\pagestyle{fancy}
\lhead{TCSS 544 - Spring 2016}
\chead{\textbf{Clustering and \\ Dimensional Reduction Project}}
\rhead{J. McFadden}
\headsep = 22pt 
\headheight = 15pt

%\title{Comparison of Cloud NoSQL Software, Systems and Platforms}
%\date{June 4, 2016}
%\author{J. McFadden \\ Univ. of Washington: Tacoma \\ Tacoma, WA \\ mcfaddja@uw.edu \and Y. Tamta \\ Univ. of Washington: Tacoma \\ Tacoma, WA \\ yashaswitamta@gmail.com \and J. N. Gandhi \\ Univ. of Washington: Tacoma \\ Tacoma, WA \\ jugalg@uw.edu }

%\doublespacing
%\singlespacing

% BEGIN PRE-AMBLE


% Setup equation numbering 
\numberwithin{equation}{section} 

%Equation Numbering Shortcut Commands
\newcommand{\numbch}[1]{\setcounter{section}{#1} \setcounter{equation}{0}}
\newcommand{\numbpr}[1]{\setcounter{subsection}{#1} \setcounter{equation}{0}}
\newcommand{\note}{\textbf{NOTE:  }}

%Formatting shortcut commands
\newcommand{\chap}[1]{\begin{center}\begin{Large}\textbf{\underline{#1}}\end{Large}\end{center}}
\newcommand{\prob}[1]{\textbf{\underline{Problem #1):}}}
\newcommand{\sol}[1]{\textbf{\underline{Solution #1):}}}
\newcommand{\MMA}{\emph{Mathematica }}

%Text Shortcut Command
\newcommand{\s}[1]{\emph{Side #1}}

% Math shortcut commands
\newcommand{\dd}[2]{\frac{d #1}{d #2}}
\newcommand{\ddn}[3]{\frac{d^{#1} #2}{d #3^{#1}}}
%\newcommand{\dd}[2]{\frac{\textrm{d} #1}{\textrm{d} #2}}
%\newcommand{\ddn}[3]{\frac{\textrm{d}^{#1} #2}{\textrm{d} #3^{#1}}}
\newcommand{\pd}[2]{\frac{\partial #1}{\partial #2}}
\newcommand{\pdn}[3]{\frac{\partial^{#1} #2}{\partial #3^{#1}}}
\newcommand{\infint}{\int_{-\infty}^\infty}
\newcommand{\infiint}{\iint_{-\infty}^\infty}
\newcommand{\infiiint}{\iiint_{-\infty}^\infty}
\newcommand{\dint}[2]{\int_{#1}^{#2}}
\newcommand{\intdd}[1]{\textrm{d}#1}
\newcommand{\intddd}[1]{\textrm{d}#1}
\newcommand{\R}{\mathbb{R}}
\newcommand{\N}{\mathbb{N}}
\newcommand{\Z}{\mathbb{Z}}
%\newcommand{\mat}[1]{\overleftrightarrow{\mathbf{#1}}}
%\newcommand{\mat}[1]{\bar{\bar{\mathbf{#1}}}}
\newcommand{\mat}[1]{\overline{\overline{\mathbf{#1}}}}

%Math Text
\newcommand{\rect}{\text{ rect}}
\newcommand{\csch}{\text{ csch}}

%Physics Shortcut Commands
\newcommand{\h}{\mathcal{H}}


%MRI Stuff Shortcut Commands
\newcommand{\tno}{t_{n}}
\newcommand{\tn}[1]{t_{n#1}}
\newcommand{\Mno}{\vec{M}^{\left( n \right)}}
\newcommand{\Mn}[1]{\vec{M}^{\left( n #1 \right)}}
\newcommand{\Mnto}[1]{\vec{M}^{(n)} \left( t_{n} #1 \right)}
\newcommand{\Mnt}[2]{\vec{M}^{(n #1)} \left( t_{n #1} #2 \right)}
\newcommand{\rot}[2]{\mat{R}_{#1} \left( #2 \right)}
\newcommand{\DnMat}[2]{\mat{D} \left( t_{n #1} #2 \right)}
\newcommand{\rotINV}[2]{\mat{R}^{-1}_{#1} \left( #2 \right)}
\newcommand{\DnMatINV}[2]{\mat{D}^{-1} \left( t_{n #1} #2 \right)}
\newcommand{\betaNt}[2]{\beta \left( t_{n #1} #2 \right)}
\newcommand{\TR}{\textrm{TR}}


% Math formatting commands
\newcommand{\stack}[2]{\stackrel{\mathclap{\normalfont\mbox{#1}}}{#2}}


% END PRE-AMBLE
\renewcommand{\baselinestretch}{1.0} 


\begin{document}
\thispagestyle{empty}

%\maketitle
%\begin{titlepage}
%	\centering
$\phantom{1}$ \\
\vspace{0.2in}
\begin{center}
	\Huge{\textbf{\underline{Comparison of NoSQL on the Cloud}}} \\
	\vspace{0.10in}
	\Huge{\textbf{\underline{Software, Systems and Platforms}}} \\
	\vspace{0.20in}
	\begin{multicols}{3}
		\Large{\textbf{J. McFadden}} \\
		\vspace{0.1in} 
		\normalsize{Univ. of Washington: Tacoma} \\ \small{Tacoma, WA} \\ 
		\vspace{0.05in}
		\large{\emph{mcfaddja@uw.edu}}  \\
		
		\Large{\textbf{Y. Tamta}} \\ 
		\vspace{0.1in} 
		\normalsize{Univ. of Washington: Tacoma} \\ \small{Tacoma, WA} \\ 
		\vspace{0.05in} 
		\large{\emph{yashaswitamta@gmail.com}} \\
		
        		\Large{\textbf{J. N. Gandhi}} \\
		\vspace{0.1in} 
		\normalsize{Univ. of Washington: Tacoma} \\ \small{Tacoma, WA} \\ 
		\vspace{0.05in} 
		\large{\emph{jugalg@uw.edu}}
	\end{multicols}
	\vspace{0.3in}
	\Large{\textbf{\underline{April 11, 2017}}} \\
	\vspace{0.1in}
	\small{\emph{Project coordinator indicated by \textbf{*}}}
\end{center}

\vspace{0.9in}
%\end{titlepage}




\begin{multicols}{2}
\begin{flushleft}


\section*{Abstract}
\begin{spacing}{1.5}
The software/systems chosen for comparison in this project are two different NoSQL database system.  These systems will be deployed/run/operated in several different ways.  These include \emph{SaaS}\footnote{\textbf{SaaS :} Software as a service.} implementations, \emph{containerized} implementations, and \emph{native installations}.  The goal of the project is to understand the performance characteristics of each deployment method \emph{\textbf{and}} to quantify the costs of each deployment method.  These costs will be calculated based on the hourly cost to operate, the initial time \& costs required for setup, and the maintenance requirement of a deployment.  Additionally, performance of the systems and deployments will be measured using the time required to carry out various database operations, under a set of several different conditions, as well as the CPU, memory, and network loads imposed by the various deployments under the same set of conditions.
\end{spacing}




%\section{Software, Platforms, \& Systems}
\section{Systems and Platforms}
\begin{spacing}{1.5}
We will be using two NoSQL database software packages.  The first software package is \textbf{DynamoDB} from Amazon Web Services (\emph{AWS}), while the second software package will be \textbf{Cassandra}, an open-source NoSQL database software package.  These software packages will be deployed using several different systems and platforms, as described below.
\end{spacing}



\subsection{Systems}
\begin{spacing}{1.5}
This project will use four different systems for running these software packages.  These systems range from hosted \emph{SaaS} through various degrees of virtualization and then, finally, to non-virtualized machines.  These systems are 

\begin{itemize}
	\item \textbf{AWS \emph{SaaS} system(s)}
	\item \textbf{Virtualization using Docker containers}
	\item \textbf{Virtualization on AWS's EC2 VMs}
	\item \textbf{Dedicated, non-virtualized machines}
\end{itemize}

These four systems will be deployed on to several different platforms.  Thus, we will now move on to describing the platforms that will be used to deploy these systems.
\end{spacing}




\subsection{Platforms}
\begin{spacing}{1.5}
We have chosen three different platforms on which to deploy the software packages, with the platforms spanning the range of cloud service paradigms from \emph{SaaS} to \emph{PaaS}\footnote{\textbf{PaaS :} Platform as a service.} to \emph{IaaS}\footnote{\textbf{IaaS :} Infrastructure as a service.}.  These platforms are 

\begin{itemize}
	\item \textbf{AWS's DynamoDB Service (}\emph{SaaS}\textbf{)}
	\item \textbf{Containerized implementations (}using \emph{Docker}\textbf{)} running on
	\begin{itemize}
		\item \textbf{\emph{AWS's Container Service} (}\emph{PaaS}\textbf{)}
		\item \textbf{\emph{AWS EC2 Machines running the docker run-time in Linux} (}hybrid \emph{Pass/IaaS}\textbf{)}
	\end{itemize}
	\item \textbf{AWS EC2 Machines running native installations of the software in Linux (}\emph{IaaS}\textbf{)}
\end{itemize}
\end{spacing}



\section{Deployment}




















\end{flushleft}
\end{multicols}












































%\end{flushleft}
\end{document}